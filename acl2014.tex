\documentclass[11pt]{article}
\usepackage{acl2014}
\usepackage{times}
\usepackage{url}
\usepackage{latexsym}
\usepackage{color}


\title{Wildcards and Morphological Inflections for the Google Books Ngram Corpus}

\author{Jason Mann, David Zhang, Lucille Yang, Slav Petrov and Dipanjan Das\\
	Google Inc. \\
	{\tt jcm2207@columbia.edu, dzhang21@gmail.com, ly77@cornell.edu}\\
	{\tt \{slav,dipanjand\}@google.com}}

\date{}

\begin{document}
\maketitle

\begin{abstract}

We present a new edition of the Google Books Ngram Viewer, which plots
the frequency with which words and phrases were used over the last five
centuries; it's data encompasses 23\% of the world's published books.
The new edition adds three tools for more powerful search: wildcards,
morphological inflections, and case insensitivity. These additions allow
for the discovery of previously unknown patterns in the Ngram Viewer data,
and further facilitate the study of linguistic trends in printed text.

\end{abstract}

\section{Introduction}

The Google Ngram Viewer \url{http://books.google.com/ngrams} and its corresponding Google Books Corpus \cite{culturomics}, since their release in 2010 have assisted in the analysis of cultural and linguistics trends through five centuries of data in eight languages. We present an updated version of the Viewer which introduces several new features. 

First, users can replace one term or tag in their queries with a placeholder symbol (wildcard, henceforth), which will return the ten most frequent replacements in the Google Books corpus for the specific year range entered. Second, by adding a specific tag to any word in a query, morphological inflections (or variants) will be returned. Finally, the new interface has an additional option to allow for multiple capitalization styles. In addition, this demonstration presents an overhaul of the Viewer's user interface, with interactive features that allow for easier management of the increase in data points returned.

\vspace{1em}
\textcolor{red}{Mention related and prior work here.}
\vspace{1em}

In the following section, we provide an overview of our system's structure. We detail interesting use cases in section~\ref{sec:usecases}, which were difficult (or even impossible) to search in the previous versions of the NGram viewer that did not handle wildcards in the search queries. Additionally, we detail how the two other aforementioned features introduced in this demonstration paper result in interesting retrieval results. Beyond specific searches, we envision the new functionality of the tool uncovering trends and patterns not readily apparent in the data.

\section{System Overview}

In this section we present an overview of the Google NGram Viewer backend. Before going into the details, we first describe the corpora on which users of this tool can issue queries.

\subsection{Ngram Corpus}
The Google Books Ngram Corpus is available at \url{http://storage.googleapis.com/books/ngrams/books/datasetsv2.html}. The corpus provides ngram counts for eight different languages over more than 500 years; additionally, the english corpus is split further into British English and Fiction to aid domain restriction. This corpus is a subset of all the books digitized at Google, and represents more than 6\% of all publicized texts in its newest edition. The differences between the first and second versions of the corpus are discussed at length in \newcite{lin2012syntactic} and the work in this demonstration is limited to the latest 2012 version.
\textcolor{red}{\bf Were there any improvements past this?}

\begin{table*}
\centering
\setlength{\tabcolsep}{3pt}
\begin{table}[h]
\begin{tabular}{|l|l|l|l|l|}
\hline
\multicolumn{5}{|l|}{} \\ \hline
   &    &    &    &    \\ \hline
   &    &    &    &    \\ \hline
   &    &    &    &    \\ \hline
\end{tabular}
\end{table}
\end{table*}

\subsection{Wildcards}
We support the use of wildcards by utilizing an additional database that stores the most frequent replacements of queries to the ngram corpus. This wildcard database is precompiled from the existing corpus by first replacing each word or tag in the ngram with the wildcard character `*'
We support the use of wildcards by using an additional database (or table) that stores the most frequent replacements of search queries that could feature a wildcard. Next, describe the process in an algorithmic fashion, without going into particular implementation details.

\subsection{Morphological Inflections}
Inflections of words in search queries are handled using the Google Search interface that can provide morphological variations of words for different syntactic categories (provide footnote about the ?define? keyword in Google Search). Provide more details as to how this works.

\subsection{Capitalization}

\subsection{User Interface}
\textcolor{red}{\bf Should we talk about the interface?}


\section{Use Cases}
\label{sec:usecases}
We present multiple use cases that can be captured using the several features that we have presented in this paper. First, we show some examples of each of these individual features; next, we present some example queries that combine queries that use syntactic annotations (Lin et al, 2012) and the current additions to exhibit the type of results that the NGram Viewer can retrieve.


\section{Conclusions}
We have presented a new version of the NGram Viewer with some new functionality. With the introduction of these new features, users can perform more powerful searches that show trends which were not possible to extract from earlier versions of the tool. 

\textcolor{red}{We can cite examples from the media where this has been mentioned, and also show examples from several blog posts/entries from the internet:
http://sciencerefinery.com/2013/10/28/google-ngram-viewer-now-more-powerful-than-ever/
http://www.devingriffiths.com/google-books/google-n-gram-studies/
http://languagelog.ldc.upenn.edu/nll/?p=8472
http://www.textualscholarship.nl/?p=14051}



\bibliographystyle{acl}
\bibliography{acl2014}
\end{document}